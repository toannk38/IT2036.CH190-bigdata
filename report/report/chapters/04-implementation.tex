\chapter{Triển khai hệ thống}

\section{Cấu trúc mã nguồn}

Dự án được tổ chức theo kiến trúc microservices với cấu trúc thư mục rõ ràng:

\begin{lstlisting}[language=bash]
src/
├── collectors/          # Thu thập dữ liệu
│   ├── price_collector.py
│   └── news_collector.py
├── consumers/           # Kafka consumers
│   └── kafka_consumer.py
├── engines/            # AI/ML và LLM engines
│   ├── aiml_engine.py
│   └── llm_engine.py
└── services/           # API và aggregation
    ├── api_service.py
    └── aggregation_service.py
\end{lstlisting}

\section{Cấu hình Docker}

Hệ thống sử dụng Docker Compose để orchestrate các services:

\begin{itemize}
\item \textbf{Kafka}: Message streaming platform
\item \textbf{MongoDB}: NoSQL database cho việc lưu trữ
\item \textbf{Airflow}: Workflow orchestration
\item \textbf{API Service}: RESTful API endpoints
\end{itemize}

\section{Price Collector (5-minute interval)}

Price Collector thu thập dữ liệu giá cổ phiếu từ vnstock API mỗi 5 phút:

\begin{lstlisting}[language=Python]
# Placeholder for Price Collector implementation
class PriceCollector:
    def collect_prices(self, symbols):
        # Collect stock prices from vnstock API
        pass
\end{lstlisting}

\section{News Collector (30-minute interval)}

News Collector thu thập tin tức tài chính mỗi 30 phút:

\begin{lstlisting}[language=Python]
# Placeholder for News Collector implementation
class NewsCollector:
    def collect_news(self):
        # Collect financial news
        pass
\end{lstlisting}

\section{Analysis Engines}

\subsection{AI/ML Engine}
Engine phân tích định lượng sử dụng machine learning algorithms.

\subsection{LLM Engine}
Engine phân tích sentiment và tóm tắt tin tức sử dụng Large Language Models.